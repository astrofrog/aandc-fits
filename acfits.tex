%% This is file `elsarticle-template-2-harv.tex',
%%
%% Copyright 2009 Elsevier Ltd
%%
%% This file is part of the 'Elsarticle Bundle'.
%% ---------------------------------------------
%%
%% It may be distributed under the conditions of the LaTeX Project Public
%% License, either version 1.2 of this license or (at your option) any
%% later version.  The latest version of this license is in
%%    http://www.latex-project.org/lppl.txt
%% and version 1.2 or later is part of all distributions of LaTeX
%% version 1999/12/01 or later.
%%
%% The list of all files belonging to the 'Elsarticle Bundle' is
%% given in the file `manifest.txt'.
%%
%% Template article for Elsevier's document class `elsarticle'
%% with harvard style bibliographic references
%%
%% $Id: elsarticle-template-2-harv.tex 155 2009-10-08 05:35:05Z rishi $
%% $URL: http://lenova.river-valley.com/svn/elsbst/trunk/elsarticle-template-2-harv.tex $
%%

%%\documentclass[preprint,authoryear,12pt]{elsarticle}

%% Use the option review to obtain double line spacing
%% \documentclass[authoryear,preprint,review,12pt]{elsarticle}

%% Use the options 1p,twocolumn; 3p; 3p,twocolumn; 5p; or 5p,twocolumn
%% for a journal layout:

%% Astronomy & Computing uses 5p
%% \documentclass[final,authoryear,5p,times]{elsarticle}
\documentclass[final,authoryear,5p,times,twocolumn]{elsarticle}

%% if you use PostScript figures in your article
%% use the graphics package for simple commands
%% \usepackage{graphics}
%% or use the graphicx package for more complicated commands
% \usepackage{graphicx}
%% or use the epsfig package if you prefer to use the old commands
%% \usepackage{epsfig}

%% The amssymb package provides various useful mathematical symbols
% \usepackage{amssymb}
\usepackage{gensymb}
%% The amsthm package provides extended theorem environments
%% \usepackage{amsthm}

\usepackage[pdftex,pdfpagemode={UseOutlines},bookmarks,bookmarksopen,colorlinks,linkcolor={blue},citecolor={green},urlcolor={red}]{hyperref}
% \usepackage{hypernat}

\usepackage{breakurl}

%% The lineno packages adds line numbers. Start line numbering with
%% \begin{linenumbers}, end it with \end{linenumbers}. Or switch it on
%% for the whole article with \linenumbers after \end{frontmatter}.
%% \usepackage{lineno}

%% natbib.sty is loaded by default. However, natbib options can be
%% provided with \biboptions{...} command. Following options are
%% valid:

%%   round  -  round parentheses are used (default)
%%   square -  square brackets are used   [option]
%%   curly  -  curly braces are used      {option}
%%   angle  -  angle brackets are used    <option>
%%   semicolon  -  multiple citations separated by semi-colon (default)
%%   colon  - same as semicolon, an earlier confusion
%%   comma  -  separated by comma
%%   authoryear - selects author-year citations (default)
%%   numbers-  selects numerical citations
%%   super  -  numerical citations as superscripts
%%   sort   -  sorts multiple citations according to order in ref. list
%%   sort&compress   -  like sort, but also compresses numerical citations
%%   compress - compresses without sorting
%%   longnamesfirst  -  makes first citation full author list
%%
%% \biboptions{longnamesfirst,comma}

% \biboptions{}

\journal{Astronomy \& Computing}

%% Upright single quotes in verbatim fields make FITS header examples
%% much more readable
\usepackage{upquote}

%% For draft use color package to indicate open questions that need
%% clarification
\usepackage{color}

\begin{document}

\begin{frontmatter}

%% Title, authors and addresses

%% use the tnoteref command within \title for footnotes;
%% use the tnotetext command for the associated footnote;
%% use the fnref command within \author or \address for footnotes;
%% use the fntext command for the associated footnote;
%% use the corref command within \author for corresponding author footnotes;
%% use the cortext command for the associated footnote;
%% use the ead command for the email address,
%% and the form \ead[url] for the home page:
%%
%% \title{Title\tnoteref{label1}}
%% \tnotetext[label1]{}
%% \author{Name\corref{cor1}\fnref{label2}}
%% \ead{email address}
%% \ead[url]{home page}
%% \fntext[label2]{}
%% \cortext[cor1]{}
%% \address{Address\fnref{label3}}
%% \fntext[label3]{}

\title{The Future of Astronomical Data Formats I. Learning from FITS}

%% use optional labels to link authors explicitly to addresses:
%% \author[label1,label2]{<author name>}
%% \address[label1]{<address>}
%% \address[label2]{<address>}

\author[noao]{Brian~Thomas\corref{cor1}}
\ead{bthomas@noao.edu}
\author[cornell]{Tim~Jenness\corref{cor1}}
\ead{tjenness@cornell.edu}
\author[noao]{Frossie~Economou}
\author[stsci]{Perry~Greenfield}
\author[geminin]{Paul~Hirst}
\author[jac]{David~S.~Berry}
\author[stsci]{Erik~Bray}
\author[glasgow]{Norman~Gray}
\author[ohio]{Demitri~Muna}
\author[geminis]{James~Turner}
\author[princeton]{Miguel~de~Val-Borro}
\author[iaa]{Juande~Santander-Vela}
\author[ipac]{David~Shupe}
\author[ipac]{John~Good}
\author[ipac]{G.~Bruce~Berriman}
\author[icrar]{Slava~Kitaeff}
\author[microsoft]{Jonathan~Fay}
\author[sao]{Omar~Laurino}
\author[stsci]{Anastasia~Alexov}
\author[ipac]{Walter~Landry}
\author[nrao]{Joe~Masters}
\author[cornell]{Adam~Brazier}
\author[aifa]{Reinhold~Schaaf}
\author[uwaterloo]{Kevin~Edwards}
\author[jac]{Russell~O.~Redman}
\author[warwick]{Thomas~R.~Marsh}
\author[aip]{Ole~Streicher}
\author[noao]{Pat~Norris}
\author[ucm]{Sergio~Pascual}
\author[unsw]{Matthew~Davie}
\author[stsci]{Michael~Droettboom}
\author[mpia]{Thomas~Robitaille}
\author[iasf]{Riccardo~Campana}
\author[psu]{Alex~Hagen}
\author[mps]{Paul~Hartogh}
\author[aifa]{Dominik~Klaes}
\author[msum]{Matthew~W.~Craig}
\author[cral]{Derek~Homeier}

\cortext[cor1]{Corresponding authors}

\address[noao]{Science Data Management, National Optical Astronomy Observatory, 950 N Cherry Ave, Tucson, AZ 85719, USA}
\address[cornell]{Department of Astronomy, Cornell University, Ithaca,
  NY 14853, USA}
\address[stsci]{Space Telescope Science Institute, 3700 San Martin Drive, Baltimore, MD 21218, USA}
\address[geminin]{Gemini Observatory, 670 N.\ A`oh\=ok\=u Place, Hilo, HI 96720, USA}
\address[jac]{Joint Astronomy Centre, 660 N.\ A`oh\=ok\=u Place, Hilo, HI 96720, USA}
\address[glasgow]{SUPA School of Physics \& Astronomy, University of Glasgow, Glasgow, G12 8QQ, UK}
\address[ohio]{Department of Astronomy, The Ohio State University, Columbus, OH 43210, USA}
\address[geminis]{Gemini Observatory, Casilla 603, La Serena, Chile}
\address[princeton]{Department of Astrophysical Sciences, Princeton University, Princeton, NJ 08544, USA}
\address[iaa]{Instituto de Astrof\'isica de Andaluc\'ia, Glorieta de la Astronom\'ia s/n, E-18008, Granada, Spain}
\address[ipac]{Infrared Processing and Analysis Center, Caltech, Pasadena, CA 91125, USA}
\address[icrar]{International Centre for Radio Astronomy Research, M468, 35 Stirling Hwy, Crawley, Perth WA 6009, Australia}
\address[microsoft]{Microsoft Research, 14820 NE 36th Street, Redmond, WA 98052, USA}
\address[sao]{Smithsonian Astrophysical Observatory, 60 Garden Street, Cambridge,
MA 02138, USA}
\address[nrao]{National Radio Astronomy Observatory, 520 Edgemont Road,
Charlottesville, VA 22903, USA}
\address[aifa]{Argelander-Institut f\"{u}r Astronomie, Universit\"{a}t Bonn, Auf dem H\"{u}gel 71, 53121 Bonn, Germany}
\address[uwaterloo]{Department of Physics, University of Waterloo, Waterloo, ON N2L~3G1, Canada}
\address[warwick]{Department of Physics, University of Warwick, Coventry CV4 7AL, UK}
\address[aip]{Leibniz-Institut für Astrophysik Potsdam (AIP), An der Sternwarte 16, 14482 Potsdam, Germany}
\address[ucm]{Departamento de Astrof\'{i}sica, Universidad Complutense de Madrid, 28040, Madrid, Spain}
\address[unsw]{Department of Astrophysics, School of Physics,
  University of New South Wales, Sydney, NSW 2052, Australia}
\address[mpia]{Max-Planck-Institut f\"{u}r Astronomie, K\"{o}nigstuhl 17, 69117 Heidelberg, Germany}
\address[iasf]{Institute for Space Astrophysics and Cosmic Physics, Via Piero Gobetti 101, Bologna, I-40129, Italy}
\address[psu]{Dept.\ of Astronomy and Astrophysics, The Pennsylvania
  State University, 525 Davey Lab, University Park, PA 16802, USA}
\address[mps]{Max-Planck-Institut f\"{u}r Sonnensystemforschung,
  Justus-von-Liebig-Weg 3, 37077 G\"{o}ttingen, Germany}
\address[msum]{Department of Physics and Astronomy, Minnesota State University Moorhead, 1104 7th Ave. S., Moorhead, MN 56563, USA}
\address[cral]{Centre de Recherche Astrophysique de Lyon, UMR 5574, CNRS,
  Universit\'{e} de Lyon, ENS Lyon, %% \'{E}cole Normale Sup\'{e}rieure de Lyon,
  46 All\'{e}e d'Italie, 69364 Lyon Cedex 07, France}

\begin{abstract}
%% Text of abstract

The Flexible Image Transport System (FITS) standard has been a great
boon to astronomy, allowing observatories, scientists and the public
to exchange astronomical information easily. The FITS standard,
however, is showing its age. Developed in the late 1970s, the FITS
authors made a number of implementation choices that, while common at
the time, are now seen to limit its utility with modern data. The
authors of the FITS standard could not anticipate the challenges which
we are facing today in astronomical computing. Difficulties we now
face include, but are not limited to, having to address the need to
handle an expanded range of specialized data product types (data
models), being more conducive to the networked exchange and storage of
data, handling very large datasets, and the need to capture
significantly more complex metadata and data relationships.


There are members of the community today who find some (or all) of
these limitations unworkable, and have decided to move ahead with
storing data in other formats. This reaction should be taken as a
wakeup call to the astronomy community to make changes in the FITS
standard or to see its usage fall. In this paper we detail some
selected important problems which exist within the FITS standard
today. It is not our intention here to prescribe specific remedies to
these issues; rather, we hope to call attention of the FITS and
greater astronomical computing communities to these issues in the
hopes that it will spur action to address them.

\end{abstract}

\begin{keyword}
%% keywords here, in the form: keyword \sep keyword

%% MSC codes here, in the form: \MSC code \sep code
%% or \MSC[2008] code \sep code (2000 is the default)

FITS \sep
File formats \sep
Standards

\end{keyword}

\end{frontmatter}

% \linenumbers

\newcommand{\aspconf}{ASP Conf.\ Ser}
\newcommand{\aap}{A\&A}
\newcommand{\aaps}{A\&AS}
\newcommand{\jrasc}{JRASC}
\newcommand{\qjras}{QJRAS}
\newcommand{\mnras}{MNRAS}
\newcommand{\pasp}{PASP}
\newcommand{\pasa}{PASA}
\newcommand{\apjs}{ApJS}

%% main text
\section{Introduction}


The Flexible Image Transport System standard (FITS;
\citealt{1979ipia.coll..445W,1980SPIE..264..298G,1981A&AS...44..363W,1981A&AS...44..371G} and
\citealt{2001A&A...376..359H}; and more recently, the definition of the
version 3.0 FITS standard by \citealt{2010A&A...524A..42P}) has been a
fundamental part of astronomical computing for a significant part of the
past four decades. The FITS format became the central means to store and
exchange astronomical data, and because of hard work by the FITS
community it has become a relatively easy exercise for application
writers, archivists, and end user scientists to interchange data and
work productively on many computational astronomy problems. The success
of FITS is such that it has even spread to other domains such as medical
imaging and digitizing manuscripts in the Vatican Library
\citep{2006JRASC.100..242W,2012EWASSAlle}.


Although there have been some significant changes, the FITS standard
has evolved very slowly since its genesis in the late 1970s. New types
of metadata conventions such as World Coordinate System
\citep[WCS;][]{2002A&A...395.1061G,2002A&A...395.1077C,2006A&A...446..747G}
representation and data serializations such as variable length binary
tables \citep{1995A&AS..113..159C} have been added. Nevertheless,
these changes have not been sufficient to match the greater evolution
in astronomical research over the same period of time.


Astronomical research now goes beyond the paradigm of a set of
observational data being analyzed only by the scientific team who
proposed or collected it. The community routinely combines original
observations, theoretical calculations, observations from others, and
data from archives on the internet in order to obtain better and wider
ranging scientific results. A wide variety of research projects now involve many
diverse data sets from a range of sources. Instruments in astronomy
now produce several orders of magnitude larger datasets than were common
at the time FITS was born, in some cases requiring parallelized,
distributed storage systems to provide adequate data rates
\citep{2012ASPC..461..283A}.


Astronomers have increasingly come to rely on others to write software
programs
to help process and analyze their data. Common libraries, analysis
environments, pipeline processed data, applications and services
provided by third parties form a crucial foundation for many
astronomers' toolboxes. All of this requires that the interchange of
data between different tools needs to be as automated as possible, and
that complex data models and metadata used in processing are
maintained and understood through the interchange.


These changes in research practices pose new challenges for the
21\textsuperscript{st} century. We must address the need to handle an
expanded range of specialized data product types and models, be more
conducive to the distributed exchange and storage of data, handle very
large datasets and provide a means to capture significantly more complex
metadata and data relationships.


A summary of these significant problems within the FITS standard was
presented in \citet{P90_adassxxiii}.  Already some of these limitations
have caused members of the community to seek more capable storage
formats, both in the past, such as the Starlink Hierarchical Data System
\citep[HDS;][]{1982QJRAS..23..485D}, the eXtensible Data Format
\citep[XDF;][]{2001ASPC..238..217S}, FITSML \citep{2001ASPC..238..487T}
and HDX \citep{2003ASPC..295..221G}; and in the present and future
(e.g., HDF5 \citep{2011ASPC..442...53A} and NDF \citep{2014Jenness}).
There are other popular file formats amongst the
radio and (sub-)millimeter astronomy community such as the Continuum and
Line Analysis Single-dish Software (CLASS) data format associated with
the Grenoble Image and Line Data Analysis Software (GILDAS) tools
({\href{http://ascl.net/1305.010}{ascl:1305.010}). Although this file
format does not have a public specification,  there are open-source
spectroscopic software packages like \texttt{PySpecKit}
({\href{http://ascl.net/1109.001}{ascl:1109.001}) that support certain
versions of the data format.  Given the large amount of available
storage formats, there is certainly a possibility that the use of FITS
will fall in favor of other scientific data formats should it not adapt
to these new challenges.


Astronomers can now simulate accurately problems that involve fluid
flows like the collapse of molecular clouds to form stars and the
formation of cosmological large scale structures using numerical methods
on parallel supercomputers.  The results of these massive computer
simulations that use either Eulerian methods like in adaptive mesh
refinement (AMR) codes or Lagrangian methods like smoothed-particle
hydrodynamics (SPH) codes can be stored in a variety of data formats.
Certain packages like the \texttt{yt} toolkit provide a consistent
interface to analyzing and visualizing astrophysical simulation data
formats such as HDF5 and other native formats found in some popular AMR
codes \citep{2011ApJS..192....9T}. FITS output may be supported in some
simulation codes to export the data to the final synthetic images or
spectra to allow direct comparison with observational data, although
FITS files are not generally used to store gridded or particle data from
simulations.


It is our intended goal in this paper to highlight some selected,
important, problems which exist in the FITS core standard today.
By intent this paper does not prescribe specific solutions to these issues;
rather, we hope to bring awareness of these issues to the FITS and greater
astronomical computing communities in the hopes that
it will spur action to resolve them.

\section{Deficiencies of FITS for Modern Astronomical Research}

As technologies and research techniques in astronomy have evolved, FITS
has not kept pace. As a result, gaps between FITS utility and the
needs of the research community have opened up and widened over time. In
this part of the paper, we detail many issues, or deficiencies of FITS,
which we see in this regard.


The deficiencies of FITS can be categorized into several groups.
These may be summarized as poor support for information
interchange (section \ref{section_poor_exchange}),
missing critical data models (section \ref{section_crit_data_models}), inflexibility in
representing both metadata (section \ref{section_inflex_represent}) and data and inadequate
support for large and/or distributed data (section \ref{section_poor_large_data_support}). We address each of these below.

\subsection{Poor support for information interchange}
\label{section_poor_exchange}

FITS originated as a delivery format for observatory data. It was the format
of choice when transporting data between different data reduction
environments such as IRAF (\href{http://ascl.net/9911.002}{ascl:9911.002}),
Starlink (\href{http://ascl.net/1110.012}{ascl:1110.012}), AIPS
(\href{http://ascl.net/9911.003}{ascl:9911.003}) and MIDAS
(\href{http://ascl.net/1302.017}{ascl:1302.017}).


In principle, FITS promotes interchange through its simple and easily
understood format which holds its information in various levels of groupings
of metadata and data blocks. Metadata are captured via key-value pairs which
are in turn grouped into FITS headers. The first header is denoted as the
`primary' header and subsequent headers known as `extensions'. Headers may or
may not be then grouped with data blocks.  An example primary FITS header
appears in Fig.~\ref{fig:fitshead}.


\begin{figure*}
\begin{minipage}{\textwidth}
\begin{center}
\begin{verbatim}
SIMPLE  =                    T / Standard FITS format
BITPIX  =                  -32 / 32 bit IEEE floating point numbers
NAXIS   =                    3 / Number of axes
NAXIS1  =                  800 /
NAXIS2  =                  800 /
NAXIS3  =                    4 /
EXTEND  =                    T / There may be standard extensions
ATODGAIN=             7.000000 / Analog to Digital Gain (Electrons/DN)
RNOISE  =             1.010153 / Readout Noise (DN)
EPOCH   =       49740.82869315 / exposure average time (Modified Julian Date)
EXPTIME =          2500.000000 / exposure duration (seconds)--calculated
EXP0    =          1300.000000 / weighted average initial exposure time
RSDPFILL=                 -250 / bad data fill value for calibrated images
SATURATE=                10237 / Data value at which saturation occurs
TEMP    =                    0 / Temperature (0=cold, 1=warm)
FILTNAM1= 'F555W             ' / first filter name
HSTPHOT =                    T / Preprocessed by HSTphot/mask
END
\end{verbatim}
\end{center}
\caption{Representative simple primary header of a FITS file showing an assortment of FITS keywords and their associated values. Bytes which contain data may or may not follow the `END' keyword of the header.}
\label{fig:fitshead}
\end{minipage}
\end{figure*}


This simple arrangement of information can satisfy many use cases for
transport, however, requirements for interchange have evolved. Effective
interchange, as we shall illustrate, now includes things like the ability to
declare models for use in higher level processing, validation of models within
the file and, at the most basic level, the ability to declare which version of
the FITS serialization is being used.


It is not as if we have not needed these things until now. In fact,
parts of the astronomical community have designed data formats to
satisfy some of these requirements. The Astronomical Data Center (ADC) XDF
format, the Low-Frequency Array for Radio Astronomy (LOFAR) HDF5 data
model \citep{2012ASPC..461..283A}, and Starlink's NDF
\citep{1988STARB...2...11C,1993ASPC...52..229W,P91_adassxxiii} all
serve as examples in this regard. XDF was created primarily to support
the development of FITSML -- an XML version of the FITS data model which
could use XML schema for validation. NDF was developed in the late
1980s as a means of organizing the hierarchical structures that were
available via the Starlink HDS format when it became apparent that
arbitrary hierarchies could lead to chaos and lack of ability for
applications to interoperate \citep{2014Jenness}.
HDX \citep{2003ASPC..295..221G} was developed around 2002 as a flexible
way of layering high-level data structures, presented as a virtual XML
DOM, atop otherwise unstructured external data stores; this was in
turn used to develop Starlink's NDX framework, which (amongst other
things) allowed FITS files to be viewed and manipulated using the
concepts of the NDF format.
HDF5 was chosen to accommodate LOFAR's
exceptional high data rates, 6-dimensional data complexity, distributed
data processing and I/O parallelization needs.

\subsubsection{Lack of serialization/format versioning}


There is no standard means which allows the FITS file to
communicate the formatting version it conforms to.  Consider the example primary header
in Fig.~\ref{fig:fitshead}, the only keyword which implies any type of
format is SIMPLE which is set to `T', or true. The comment indicates that
the file conforms to ``Standard FITS format'', but what indeed is that `Standard'?
We cannot know if the implied standard is actually 3.0 or some earlier (or later!)
version.


The designers and maintainers of FITS have espoused that a file is
``once FITS, forever FITS'' \citep[see e.g.,][]{1988A&AS...73..359G,1993FITS1}.
Certainly some in the community see this as a strength for the format
as it appears to promote long term stability and ``archivability'' of FITS
data \citep{2012EWASSAlle,2012LOC}. The assumption that FITS has not changed
is however, false. Accounting may differ, but there have been at least three
recognized named versions of FITS. These include the first, or `basic' FITS
version \citep{1979ipia.coll..445W}, the NOST version of FITS
\citep{2001A&A...376..359H} and the current version 3.0 \citep{2010A&A...524A..42P}.
Indeed, for the extreme doubters, why do we name a version `3.0' if there
were not two earlier recognized versions? It is clear then that the format
has not been static.  Even so, the assumption that the FITS format is static
has negative consequences for its further development and its use.


The first impact is that by omitting the ability to version the file no
FITS reader knows what version of the format it is reading. The usual
presumption made by software developers is that the latest version (v3.0)
is being read, however, there are still many data which predate
this version of FITS serialization making this supposition sometimes
incorrect.


The lack of versioning also limits the ability of our community to
move forward constructively with developing new FITS versions. Because
we must maintain slavish conformance to older format conventions, it
means that we accrete over time more and more ``design rules'' (or
limitations, if you will) which limit our ability to implement new and
needed features. Consider the awkward way, for example, in which the
\texttt{HIERARCH} convention was implemented \citep{2009Wic}.
In order to provide the needed functionality, the designers
were forced into utilizing \texttt{COMMENT} keywords for non-comment
functionality. This behavior by the community will only increase,
fragmenting the FITS format through inaction rather than deliberate
design.


Both of these issues needlessly drive greater and greater complexity
in implementation of FITS parsers. Furthermore, lack of versioning
arguably works counter to the purpose of archivability of the files. 
Future archivists will certainly want to know what version of the
format they are dealing with without having to guess from ancillary
evidence such as the presence of certain keywords, date of the file
creation and so on.


Versioning should be made part of the FITS standard in order to
address these problems. We cannot simply solve the lack of versioning
by adding a convention; conventions are not required to be implemented
and this is fundamental metadata about the serialization of FITS
itself. Furthermore, without the enforcement of being part of the
standard, versioning is unlikely to be implemented where it is needed
most, in the generation of new FITS files. Since any group can define
and use their own conventions to store real, meaningful data, it is
almost certain that since these conventions are not defined as part of
the file format (or even in a central location), at some time in the
future these definitions, along with the usefulness of the data, will
be lost.


\subsubsection{Inadequate declaration and validation of content meaning}


Related to, but separate from the lack of versioning of the
serialization, is the lack of ability to declare the presence of data
models and their associated semantic meaning.  By `data model' we
adopt the following meaning:


\begin{quote}
``a description of the objects represented by a computer system
together with their properties and relationships; these are typically
`real world' objects such as products, suppliers, customers, and
orders.\footnote{Definition adopted from Wikipedia, see http://en.wikipedia.org/wiki/Data\_model}
''
\end{quote}


Of course, objects in astronomy are more likely to involve things like
observations, instruments, celestial coordinates and actual astronomical
objects such as stars. Likely properties one will encounter in a FITS
file include things like observational parameters (start/end times),
astronomical coordinates, name and properties of the observing
instrumentation, and so forth. In FITS-speak, we can say that any FITS
keyword outside those defined in the FITS standard is a data model
parameter, and collections of related FITS keywords form a data model.
Ideally a data model should be associated with a given, unique,
``namespace'' so that collisions in naming of the models and requisite
parameters are avoided.


Data models can provide a standard by which information (data and metadata)
in the file may be semantically and syntactically validated in software.
Questions such as ``are all of the required metadata/data structures present
in the file?'' (e.g., all of
the needed keywords occur in the correct places in the file) and ``are
there any non-normative values in the file?'' (all metadata/data values
are within expected bounds) are both questions answered by syntactic
``validation'', the conformance of information in the file to one or more
declared data models.  The question of ``how do these data (inter)relate
with other data'' (e.g., can named structures in the file be associated
in some manner with others in another file/extension?) is one of semantic
validation. By confirming that the file is 'valid' in both senses, we may
link the data model to the information in the file, and hence answer the
fundamental question ``what does this data you gave me represent?'' (e.g., lists
of stars, tables of galaxies, images of dust clouds, etc).
It is important to note that all of these questions are critical to
consumers of the file.


There is already evidence that the FITS community values and needs shared
data models. There are many examples. WCS and some other FITS conventions
such as OIFITS
\citep{2006SPIE.6268E.106T}, MBFITS \citep{2006A&A...454L..25M},
PSRFITS \citep{2004PASA...21..302H},
SDFITS \citep{2000ASPC..216..243G} and FITS-IDI \citep{2011AIPS114}
are data models. The declaration of keyword
dictionaries\footnote{Some collected data dictionaries with FITS
keywords may be seen at the GSFC FITS site, see
\url{http://fits.gsfc.nasa.gov/fits\_dictionary.html}} is also essentially
an act of declaring one or more data model(s).


All of these data models also imply an associated ``namespace'' which is
a means of declaring the origin of the data model so that we may
disambiguate and/or associate declared properties between models.
There are common problems which namespaced models help to solve and even
the `simple' metadata in Fig.~\ref{fig:fitshead} illustrates this.


Consider the TEMP keyword in the example. Without reading the comment
associated with it, we cannot know if this is this a temperature or perhaps
some type of temporary file or resource or something else. If it is a
temperature then what is this the temperature of? What do the values '0'
and '1' mean? Are these the only valid values for this keyword?  TEMP is
a likely keyword string to appear in other files, how do we know if the
TEMP in the other files is the same one we see in the example?


As you can see, its non-trivial matter for the machine to determine whether
these are the same properties and to know other important details for using
this information. This problem is not isolated to a solitary bit
of rogue metadata. We can ask similar questions about most of the keywords
in the example header. Namespaced data models help address these issues. With
an appropriate namespace mechanism in place, it is possible to create a
machine readable mapping between the data models so that any software program
can determine whether model1:TEMP is the same (or different) property as
model2:TEMP. A good namespacing mechanism will also provide the
software program with the means to look up the model definition (perhaps from
a remote location), so that it may know the appropriate syntactic and semantic
validation rules for the information at hand. This will allow the program to
answer the remainder of our posed questions above.


It is clear then that there is a pressing need for namespaced data models,
yet, the only way in which we can currently implement them is for a human
to inspect the file, or to write special purpose software programs targeted to
particular data models. Given the data volumes that we have in astronomy, the
later choice is clearly in the direction we should go, but is not practical
in the general case.


The writing of generalized software programs to detect any data models
present in a given FITS file is currently a hopeless act for many reasons.
First and foremost, we must recognize that there are constantly new data models
being created and modified. Some of these are documented in a human readable
fashion but there are many more models which do not even meet this standard.
Worse, due in part to the lack of good validation tools, the community has
accepted many informal variants of existing models (both documented and otherwise).
Finally, there is the possible complication of more than one data model being
fully, or partially, present within a file. Without explicit signposts for
the software to use, it is likely impossible to determine which data models
are present and map information to appropriate meaning.


A means to capture and associate the data model/namespace information with
the contents of the FITS file, in a machine readable manner, should be found
so that validation can be possible without human inspection or specifically
written software programs (similar, perhaps, in the way that JSON or XML
formats have schema). This approach has additional benefits downstream. First,
it will help to avoid misinterpretation because it is better for the creator
of the data and/or data model to provide the machine readable information rather
than a downstream programmer. Second, there is a saving in effort
in that the model is done once and need not be repeated by numerous
downstream programmers. Finally, good validation tools will allow the community
to better detect informal variant models and reject them, promoting good
practice.


\subsubsection{No support for alternative encodings}


The allowed character set for metadata and data in FITS is overly
restrictive and is limiting its application. The restrictions between
metadata and data do not differ significantly. For metadata, the
keywords which occur in the FITS header, FITS only supports the 7-bit
ASCII encoding for keyword values and comments. For data encoding,
authors may again only use 7-bit ASCII for text (string) capture in
either ASCII or binary FITS tables although the NULL character is
allowed in certain cases in binary tables.


The world of astronomy has evolved beyond capturing of scientific
information in 7-bit ASCII encoding. The FITS community has grown.
Many data are captured by instruments designed, built and run by
investigators based in non-English speaking countries and astronomical
research has grown significantly elsewhere in the world. Whether as
original observational data products, reduced data, information from new
services or in capturing theoretical data, FITS is now required to hold
data which are not exclusively originating in English-speaking countries.


The current restrictive character set is an
anachronism, particularly considering that the most common language on the planet,
Chinese, cannot be used easily in a FITS file. Forcing information
into English can easily result in loss of valuable meaning, unnecessarily
limit the audience who may use the file, or force the author to use some other
format to store their data.


It is clear to us that support for alternative encodings is needed.
Simple issues which revolve around the value of keywords like the
expression of a person's name (with accents, for example), or the
ability to use special scientific and mathematical symbols (like the
\r{a}ngstr\"{o}m symbol \r{A} or the degree symbol \degree) should be
handled. Tabular text values should similarly be allowed alternative
encodings for the same reasons.  Furthermore, while not as critical, the
FITS community should also consider the utility of allowing FITS
keywords themselves to be expressed in a broader range of characters.


\subsection{Missing critical data models}
\label{section_crit_data_models}


It is not the case that FITS is missing all needed data models. In
terms of structures which represent basic organizations of the
information used in astronomy, FITS includes such things like ``table'',
``image'' or ``data cube''. These items are really simple representations
of the data at a primitive level, and are certainly needed for basic
access to the information within the file.


Nevertheless, these structures, by themselves, do not contain much in
the way of necessary detail and semantic information which tells the
consumer exactly ``what'' it is they are actually consuming. While these
structures do provide for the capture of semantic information (such as
in the form of table column keywords) an additional data model (or
more) must still be created or implemented by the author of the FITS
file if there is a desire to convey ``what'' it is. Consider the many
types of images or tables which are possible. Semantic information, in
the form of a data model, allows the author to properly label the
information so that it can be scientifically consumed without human
interpretation. Such information includes the WCS, for example.


Our concern here is that there are either data models which are
insufficient or missing in FITS and which are very basic to scientific
(and in particular, astronomical) research. Other astronomical groups
have recognized the need for some of these models. We detail
several important missing data models in this section below.

\subsubsection{Scientific Errors}


The measurement of physical properties is fundamental to astronomy
research. It is thus ironic that FITS, which is purposely designed for
supporting astronomical research, has no standard data model for
capturing information about scientific errors.


We could easily list a great number of possible error types which
might be useful but trying to encompass all of the needs of the
community at once is likely to create an unwieldy data model. We
suggest that the community needs to provide for the most common needs,
and target that subset as a first, shared model. Earlier efforts which
might inform and help this work include local data models at sites
such as NOAO, CADC \citep{2012ASPC..461..339D} and STScI,
{\color{red} (Do we have references for NOAO and STScI error models?)}
the error
models implemented in other data formats like NDF
\citep[although see for example][]{1991StarB...8...19M}, and software
efforts underway in scientific programming communities such as AstroPy
\citep{2013A&A...558A..33A}. Each of these has valuable insight into
the requirements.


Nevertheless, we can anticipate that the following general
characteristics might be part of the model:

\begin{itemize}
\item Allow for both metadata and data to have errors.

\item Allow for extensible classification of the error type. For example,
``Gaussian'' errors are also a subclass of ``statistical'' errors.

\item Allow association of more than one error class/type per
measurement. For example, allow for both systematic and statistical
errors to be associated with each measurement.

\item Allow for additional properties to be associated with each error
class. For example, ``statistical'' errors may have assigned ``sigma''
value.
\end{itemize}

\subsubsection{Limited WCS\label{sec:wcs}}


The existing FITS WCS data models illustrate some of the limitations
associated with FITS. The ``once FITS, always FITS'' idea required that
the current WCS standards were developed as an extension of the older
AIPS standard, and so inherited many of the inherent limitations of
that system. Even so they took a long time to be agreed. They are
complex yet incomplete and inflexible. They are inadequate for many
modern telescopes, and restrict creative use of novel coordinate
transformations in subsequent data analysis. For instance, raw data
must handle more distortion issues than the FITS WCS standard
projections can handle. There are some provisions for handling more
arbitrary distortions, but they are either cumbersome or too
simple. Perhaps the biggest limitation is that different
transformations of coordinates cannot be combined in flexible
ways. The user is effectively limited to choosing only one of the solutions
available.


This is unfortunate. Not only does it reduce the range of
transformations that can be described, but it also makes it harder to
analyze the total transformation into its component parts thus making
understanding and manipulation of the total transformation harder. The
alternative approach -- a ``toolkit''-style system that creates complex
transformations by stacking simpler atomic mappings -- is usually the
most efficient representation as far as data storage is concerned
(for example AST, see below).


To illustrate the problem consider the imaging data taken by the 
Hubble Space Telescope which require multiple distortion components \citep[see e.g.,][]{2013ASPC..475...49H}.
Some are small but discontinuous. Others are linear but time varying.
There is no FITS WCS compatible solution that handles these needs well.
As another example, SCUBA-2 raw data \citep[see
e.g.,][]{2013MNRAS.430.2513H} include focal plane distortions which are
combined with other transformations but must also support the dynamic
insertion of other distortion models when a Fourier transform
spectrometer \citep{2010SPIE.7741E..67G} is placed in the beam.
{\color{red} [Other examples welcome]}.


\textit{\color{red} Do we want to comment on the WCSDEP convention proposed for
  concatenating WCS models? \url{http://tdc-www.harvard.edu/wcstools/wcsdep_20020906.pdf}}

Another case with poor support is Integral Field Unit (IFU) data.
Many of these data sets
have discontinuous WCS models. The only way to support these in FITS
now is to explicitly map each pixel to the world coordinates. Besides
being space inefficient, it is difficult to manipulate in any simple
way.


In addition to limiting the description of raw telescope data, FITS
WCS also restricts what can be done with such data during subsequent
analysis. There are many potentially interesting transformations that
would result in the final WCS being inexpressible using the
restrictive FITS model. For instance, transforming an image of an
elliptical galaxy into polar or elliptical coordinates is currently not possible. Another
case which is unworkable is an alternate coordinate system to an image to
represent the pixel coordinates of a second image covering the same
part of the sky.  These may not be common requirements, but they
illustrate the wide range of transformation that should be possible
with a flexible WCS system.


The inflexibility in the FITS solution arises from multiple issues,
but lack of namespaces is a serious barrier to providing a more
flexible solution. If one has multiple model components each with
similar parameters, how does one distinguish between them? One may use
the letter suffix, but that is also used to distinguish between
alternate WCS models. The limitation on keyword sizes presents
limitations on how many coefficients can be supported. The lack of any
explicit grouping mechanism requires complex conventions on how to
relate whole sets of keywords. With more modern structures, such
contortions and limitations are not necessary.


The reality is that to solve these problems, many software systems
have chosen alternate solutions and save their WCS information in FITS
files in other ways (or in separate files). For example, the AST
library \citep{1998ASPC..145...41W,2012ASPC..461..825B} is not subject
to these limitations, but 
is forced to use non-standard FITS keywords when serializing mappings 
to FITS files (see Fig.~\ref{fig:asthead}).


\begin{figure*}
\begin{minipage}{\textwidth}
\begin{center}
\begin{verbatim}
PLRLG_A =     5.50788096462284 / Polar longitude (rad.s)
ENDAST_K= 'SphMap  '           / End of object definition
MAPB_A  = '        '           / Second component Mapping
BEGAST_O= 'CmpMap  '           / Compound Mapping
NIN_D   =                    3 / Number of input coordinates
NOUT_B  =                    2 / Number of output coordinates
INVERT_C=                    0 / Mapping not inverted
ISA_I   = 'Mapping '           / Mapping between coordinate systems
INVA_B  =                    1 / First Mapping used in inverse direction
MAPA_C  = '        '           / First component Mapping
BEGAST_P= 'MatrixMap'          / Matrix transformation
NIN_E   =                    3 / Number of input coordinates
INVERT_D=                    1 / Mapping inverted
ISA_J   = 'Mapping '           / Mapping between coordinate systems
M0_A    =    0.426766777415161 / Forward matrix value
M1_A    =    0.699933471661958 / Forward matrix value
M2_A    =    0.572680760059142 / Forward matrix value
M3_A    =   -0.418237169285184 / Forward matrix value
\end{verbatim}
\end{center}
\caption{Example header of a representation of an AST WCS object in
  Native form.}
\label{fig:asthead}
\end{minipage}
\end{figure*}


\subsubsection{History and Provenance}


The FITS standard encourages people to store processing history
information in the header using a pseudo-comment field named
\texttt{HISTORY}. This works from the perspective of making the information
available to a sufficiently interested human (assuming that each step in the
data processing adds information to the end of the history section of the
header) but the free-form nature of the entries makes it essentially
impossible for a software system to understand what was done to the data.
This may be possible within the constraints of a single data reduction
environment but it is highly unlikely that the content of the
\texttt{HISTORY} block can be understood by any other software packages. History
needs to be treated as a first-class citizen with a standardized way of
registering important information such as the date, the software tool and
any relevant arguments or switches. A library interface should also be
encouraged to make it easy for people to add history entries.


A related issue is data provenance; that is, sufficient records of how files
were created to permit their reproduction. For a given processed data product
it is, for example, impossible to determine which data files
contributed to the creation of that product. While there is no metadata standard
for specifying this
information in output files, experimental systems have been developed which, when fully developed,
aim to offer programmatic interfaces that will
simplify recording provenance information. One such example is
Provenance Aware Service Oriented Architecture
\citep[PASOA;][]{2008IPAWMoreau,2011743Moreau}, an open source architecture
already used in fields such as aerospace engineering. In brief, when
applications are executed they produce documentation of the process recorded
in a repository of provenance records that are persisted in a database. In
astronomy, PASOA was successfully demonstrated by integrating it into the
Pegasus workflow management system for running the Montage mosaic engine
\citep{2009SCGroth}.


At the JCMT Science Archive \citep[JSA;][]{2008ASPC..394..135G} data are
created with full provenance information using the native provenance
tracking that is part of NDF \citep{2009ASPC..411..418J}. This
provenance includes every ancestor along with history information that
contributed to each ancestor. When these files are converted to FITS
for ingestion into the JSA using the CAOM-2 data model
\citep{2013ASPC..475..159R} the provenance is trimmed to include
just the immediate parent files (using \texttt{PRVnnnnn} headers)
and the observation identifiers of the root ancestor observations
(using \texttt{OBSnnnnn} headers). The full richness of the provenance
information is available in FITS binary tables but the lack of standard
leaves this information hidden from applications other than the ones
that created it originally.

Finally, astronomy may benefit from methodologies used to develop provenance
systems custom to Earth Science and remote remote sensing
\citep{2008IPAWTilmes,2008IPAWMcCann}.


\subsubsection{Data Quality}


One of the more pressing needs in our era of shared and distributed
data is the need to know which data are ``good'' or, put another way, of
sufficient quality. We are long past the era when the data volume was
so small that it is practical to download all of the possible data of
interest and examine it locally.


Some might insist that this is an easily solved problem. Simply
declare a keyword, like \texttt{DQUALITY}, and allow it to take a boolean
value. That example is an exaggeration, to be sure, but it helps to
illustrate that simplicity, normally a goal of good design, will be
too simplistic. Data quality cannot be judged on a single, or even a 
small set, of parameters. The
data which are adequate for one type of use, may be wholly inadequate
in another usage context. Consider that engineering data generally are
unsuitable for science and vice versa. Many science data may be
unsuitable for other types of science (for example, studies of sky
background vs.\ pointed source science).


A data quality model then, should be an ensemble of common statistical
measures of the type of dataset which may be used to derive
higher-level judgements of the quality/suitability of the data for
some other declared purpose. There are many higher types of data
quality models which will need be created from the lower-level
measures (image data quality, pointed catalog data quality, etc) and
from these particular, targeted, statistical measures data quality may
be judged by the dataset consumer without directly examining the data
themselves.


\subsubsection{Units}


The FITS support for units is syntactically flexible (albeit with
a few specification ambiguities which would be resolved by an explicit
grammar), but the model does not accommodate the full range of
contemporary astronomical data.  This is evident from the significant
fraction of unit strings in current databases which do not conform to
the FITS prescriptions.  The solution is not simply to expand the list
of recommended units since, as well as being slow, this fails to
distinguish between, for example, different definitions of the second,
or to communicate places where the distinction does or does not
matter.

There is also a provenance issue to defining new units, since -- to
pick an example of a unit of `Jupiter core mass', two different
groups may prefer different values for the mass, and indeed may not
regard it as simply an abbreviation for a certain number of
kilogrammes, but as a scaling mass whose value is to be fixed by a
separate, subsequent, calibration.

The purely syntactical issues surrounding unit strings are to some
extent being addressed by the IVOA's `VOUnits' work \citep{VOUnits}, but the higher
level questions -- of communicating and defining new units, of
indicating documentation, and of converting between them in a
scientifically meaningful manner -- are out of scope for that work and
should be considered by the FITS community.


\subsection{Inflexibility in organizing and representing metadata and data}
\label{section_inflex_represent}


What may be construed to be ``FITS data'' has changed significantly
since the founding of FITS. The original FITS specification mandated
only the capture of astronomical images. Almost a decade later, FITS
extensions \citep{1988A&AS...73..359G}, allowing for gathering
multiple related data structures in one FITS file, and ASCII tables
\citep{1988A&AS...73..365H} were introduced. Binary tables followed in
the next decade \citep{1995A&AS..113..159C}. Changes have also
occurred in metadata capture. Over the intervening years the FITS
community has added new metadata conventions such as \texttt{HIERARCH}
\citep{2009Wic} and \texttt{GROUPING} \citep{2007Jen}, which have
allowed for greater flexibility in capturing metadata.


This expansion is to be expected and is all to the good. Nevertheless,
as we shall illustrate below, the expanded abilities to capture
information are still insufficient relative to actual need.


\subsubsection{Constrained metadata representation}


The basic element of metadata capture in a FITS file is the FITS
``card'' which comprises a keyword name, a keyword value and a
comment. All FITS cards must contain the keyword name and keyword
value pair while comments in cards are optional. Comments may
sometimes contain information about units of the metadata value.

Metadata may comprise a rich assortment of data structures and
single-valued metadata are only the beginning. There is a need to
capture sets, lists, vectors and objects within metadata (to name only
the most basic of structures). Yet FITS cards, without additional
conventions, are only capable of capturing single, scalar keyword-value 
pairings. 

The expression of non-scalar, multi-valued keywords is difficult in 
FITS and data model designers often resort to special conventions to 
achieve this. For example, in order to hold a keyword with either a 
`set' or `list' value, a common, non-standard, convention adopted is 
to create a set of keywords sharing the same base name followed by a 
integer value which may (or may not) indicate order of the values 
(such as \texttt{ICMB001}, \texttt{ICMB002}, and so on). Another example 
is the IRAF \textit{multispec} format
\citep[see][and references therein]{1993ASPC...52..467V} which uses
this scheme to specify related world coordinate information (see 
Fig.~\ref{fig:multispec} for an example).  The AST library
\citep[][and see also \S\ref{sec:wcs}]{1998ASPC..145...41W} takes a
similar approach in converting the WCS objects into FITS headers when
the transformations are too complex to be represented by standard WCS
headers (see Fig.~\ref{fig:asthead}).

\begin{figure*}
\begin{minipage}{\textwidth}
\begin{center}
\begin{verbatim}
WAT0_001= 'system=multispec'
WAT1_001= 'wtype=multispec label=Wavelength units=Angstroms'
WAT2_001= 'wtype=multispec spec1 = "1 113 2 4955.4428886353510.055675655'
WAT2_002= '83 256 0. 23.22 31.27 1. 0. 2 4 1. 256. 4963.0163112090 5.676'
WAT2_003= '976664 -0.3191636898579552 -0.8169352858733255" spec2 ="2 112'
WAT2_004= '9.09" spec 3 = "3 111 2 5043.5"                              '
\end{verbatim}
\end{center}
\caption{Example header from an IRAF \textit{multispec} data set
  indicating the use of multi-line headers..}
\label{fig:multispec}
\end{minipage}
\end{figure*}

All of this is not to say that there are not restrictions on the expression
of scalar values in headers. Consider that FITS cards are limited to 80 
characters and FITS keyword names may be no longer than 8 characters. The 
result of these constraints is that keyword values may be no longer than 
68 characters. Of course, if you use all of the space for keyword values, 
then the comment, or keyword values longer than 68 characters will need 
another convention in order to capture it (such as creating a continuation
line in the header using the \texttt{CONTINUE} convention \citep{2007Continue}.

The use of conventions is also how support may be added for more
advanced metadata structures, such as hierarchies. There are at 
least two proposed conventions which attempt to add this feature to FITS.
The most common is the \texttt{HIERARCH} convention mentioned previously, 
that co-opts \texttt{COMMENT} keywords to realize the hierarchy. The other 
convention is the \emph{record-valued} system proposed in the FITS 
distortion paper \citep{FITSDistort} which adds special syntax to the 
content of strings. 

Let us now consider the impact of keyword name constraints. Not only
are keyword names limited to a small set of characters but keyword
names are restricted to no more than 8 characters. Often these
restrictions prevent clear labelling of the metadata element because
authors are forced to map longer, more descriptive, names into the
truncated size. Non-English authors are additionally forced to map
into the limited character set. If you doubt this leads to problems,
try the following experiment: open any non-trivial FITS file and scan
the header. Unless you are an expert in the data models present in the
file (and sometimes even if you are!) its easy to find that the
cramped names of the keywords often leads to arcane and confusing
metadata.

\textit{\color{red} Is it worth mentioning here the knock-on effect that FITS
tables can only have up to 999 columns because of the need for keywords of the
form \texttt{TFORMnnn} and the 8 character limit?}

These restrictions on the FITS card have impact on conventions with
resulting limits on the utility of any implementation. Due to the limited 
namespace and size of the keywords, different conventions often reuse 
the same keywords for different purposes. For example, the previously 
mentioned issue with the \texttt{HIERARCH} convention and its use of 
\texttt{COMMENT} keywords for non-comment purposes. As another example, 
compare the use of \texttt{PV} keywords in the products of the SCAMP tools
\citep{2006ASPC..351..112B}, used for polynomial distortion coefficients, 
to the more common \texttt{PV} keywords used in the WCS convention for generic
parameter values. These two conventions, when used in the same file,
cause ambiguity and incorrect representation of the data.


Finally, while being the least problematic of these issues, but still
important, is the problem of the 2880 character logical record. While
it is largely a low-level oddity of FITS now, it does have some small
practical impact. Consider that this formatting requirement negatively
impacts many FITS files today which have significant metadata. It
often results in odd sections of the FITS file containing large blocks
of white space. Certainly, in FITS files which have very large images
(10 MB or more) this is not a particularly pressing
problem. However, for FITS files which have small amounts of data (or
are oriented primarily to metadata transport or capture) this is
simply wasteful.


The issues with metadata expression are serious for FITS with the
problems caused by the limitations on allowed FITS keyword names and
value data structures being the most significant. Certainly, as we
have already noted, in many cases it is possible to create one or more
conventions to expand the expression of the metadata but doing so will
result in files which are unparsable by downstream readers which
do not implement these conventions. There is no apparent solution,
using a FITS convention, than can solve the limitation on the
restricted expression of keyword names.


All of these metadata limitations are contrary to both the
``interchange'' and ``archivability'' use cases for FITS. If the FITS
standard is followed with additional conventions we still need
specialized readers and/or a human to successfully parse the content
of the metadata present. If the FITS standard is rigorously followed
sans conventions, then the author is severely limited in the metadata
they may express.


Keyword names and values must be allowed to be longer, and an expanded
character set would be useful as well. FITS metadata should be able to
express, as part of the FITS standard, many more data structures as
metadata. These include, but are not limited to things such as sets,
lists and objects.


\subsubsection{Awkwardness in representing the associations present in both data and metadata}


As data acquisition and data reduction systems have become more
complex there has been a move to storing multiple image data
components in extensions within a single FITS file. The FITS extension
mechanism provides a scheme for having multiple images but, as noted
in \citet{2003ASSL..285...71G}, in essentially a flat structure
without hierarchy or inheritance. If you have 12 images in the file
there is no way of indicating that three of them are data, three are an error
and three are a quality mask. Indeed, there is no way of specifying which
triplets are related. You can use the \texttt{EXTNAME} header to
indicate relationships but this relies on convention and string
parsing rather than being a standard part of the format.


The conversion of NDF format files to FITS and back to NDF
\citep{SUN55,1997STARB..19...14C}, has demonstrated that you can
represent a hierarchical data grouping in the FITS multi-extension
format, but this is done using EXTNAME conventions combined with
headers representing the extension level in the hierarchy and the type
of component and so is not understood by other FITS tools. A
standardised way of specifying relationships between extensions would
be extremely valuable to data and application interoperability.


\subsubsection{No support for declaring byte order}


The original FITS standard specification \citep{1981A&AS...44..363W}
requires that a series of consecutive bytes in multi-byte data items is
stored in order of decreasing significance (known as big endian format).
Sometimes the byte order needs to be checked and swapped to the native byte
order in systems that use the little endian format when non-native data
formats are not supported.  This is the case in some implementations of FITS
readers that do not use the \texttt{cfitsio} library
(\href{http://ascl.net/1010.001}{ascl:1010.001}) and which use C
routines to implement other scientific capabilities.  Programmers on little
endian platforms who work with large data volumes may find this limitation
results in a significant performance cost as marshalling data to and from
the FITS big endian ordering will be required.  This is a frequent problem
for astronomical programs. Little endianness is found on x86 and x86-64
processors that are commonly used in universities and research laboratories.


The inability to specify the byte order will obviously result in an infrequent
need to byte swap data. In most cases, this is not a significant
problem or impact on performance for modern software systems and
can be discounted. There is however, another, more significant issue
tied to this limitation. The ability to wrap/translate existing data products 
into FITS files, without reprocessing them to the specified
byte-order in the FITS standard, is important. From
the perspective of an archivist with the responsibility of preserving the
records of astronomical observations, the less the data are altered, the
more efficient and reliable the archival data management will be.


Other data formats used to store scientific binary data such as HDF5 or
netCDF are able to declare the byte ordering by using a data type flag and
the software libraries will perform the byte order conversion if required.
Thus, we suggest that FITS should be able to declare the byte ordering of
data by using a FITS keyword defined in the standard dictionary.


\subsection{Inadequate support for large or distributed data}
\label{section_poor_large_data_support}


At the time when FITS was developed, the primary media used for
archiving and transporting the data was a tape. A magnetic tape is
unlike a hard-drive in that it is a serial access device.  The concept
of serial accessing the data was naturally adopted for the FITS data
model.  Although tapes are still widely used for archiving data,
such a mode is no longer commonly available as the files are usually
transferred to a hard-drive before being accessed. What is more, such
serial nature of FITS has become a significant bottleneck when it comes
to large data sets.

A range of new instruments, especially in radio astronomy (ASKAP,
ALMA, MWA, LOFAR, eVLA and SKA) is producing (or will be producing)
spectral-imaging data-cubes of unprecedented volumes in the order of
hundreds of petabytes. Individual data sets can be as large as tens of
terabytes.  Servicing such data as images in FITS is going to encounter
significant performance fallbacks due to inability of FITS to provide
acceptable performance for data-cubes large than just a few gigabytes.

\subsubsection{No support for maintaining the integrity of distributed data sets}

One possible solution is to generate many small related datasets. However
the problem is that FITS does not have any mechanism to maintain the
integrity of such distributed datasets on its own. Any integrity
maintenance would be a custom solution that is likely to be
incompatible with other systems when the data are moved around.

\subsubsection{No support for parallel write/read operations on HPC platforms}

Large datasets also require parallel read/write operations to be processed on parallel high performance
computers. FITS can not support optimisation for parallel read/write operations. This has been
the driving factor for LOFAR to invest a significant effort into development of a new format using
HDF5  \citep{2012ASPC..461..283A}.

\subsubsection{No support for streaming imaging data}

Servicing large data sets to the end user also requires significant
rethinking of the framework of large data sets. FITS can only support a
cutout framework.  However, it is unreasonable that a cutout framework
can be of use for visual exploration of multi-petabyte imaging data.

A reasonable framework should be able to support multiple representations
of the same data (e.g., multiple resolutions or fidelities), preferably
from a single master file. It should also be possible to stream the data
progressively to the end-user, displaying an image as soon
as the first data become available.

Contemporary approaches widely adopted in other areas use encoded data
(e.g., wavelet transforms) to achieve such functionality \citep{2003SPIE.5150..791T}. FITS is unable to support
a comprehensive framework for encoded data, which has led to it
falling behind any modern imaging format \citep[see e.g.,][]{2013arXiv1307.5123K}.

\subsubsection{Insufficient support for compression}

Compression is going to be compulsory for large datasets as data volumes
are directly translatable to dollars. For projects like SKA with many
exabytes planned to be collected, volumes of data will be translated into
many millions of dollars.

Currently the \texttt{cfitsio} library supports several types of image
compression by convention \citep[see
e.g.,][\href{http://ascl.net/1010.002}{ascl:1010.002}]{2000ASPC..216..551P,2007ASPC..376..483S,2009PASP..121..414P}.
Binary table extensions are used to contain the encoded data, though an
image access interface is used to hide this peculiarity due to the
limitations of FITS. Obviously interoperability is lost in this case as
no other library can read such an image without also adopting the
convention.

\subsubsection{No support for capturing indeterminately sized data sets via streaming}


Frequently there is a need to store data from an instrument or remote
site that is being transmitted over a network. It is common that when
the transfer begins the final size of the data set is not known. Those
using FITS have handled this by writing such data to a file without
specifying the size of the last dimension in an image or table, and
when the stream is completed, the header is appropriately updated.


Nevertheless, there are applications for which one would like to
access all the other information before the file is complete. This may
be to integrate the data that are being read out, or to monitor
metadata. A library supporting the data format should support such
usage.


\subsubsection{No support for declaration of virtual and distributed components}


When FITS was created, the `file' (bytes stored on durable physical
medium such as spinning disk or magnetic tape) was more or less the
only way to store and transfer data. The networked solutions which we
enjoy today were absent from the world of astronomy and storage of
astronomical data in databases was unusual. Code was run locally by
experts and the results, if shared at all, were usually only reported
in published papers. In the intervening years, computer and
information technologies have evolved and broadened; we now enjoy many
new means of accessing, providing and storing data. FITS should join
this revolution.


We should start to consider thinking of FITS as a `container' of
astronomical information which is not necessarily a file. Is there
any reason to prevent our FITS `file' from overlaying a portion of a
database? Why not allow FITS to be a wrapper about bytes held within a
distributed mass store such as
iRODS\footnote{\url{http://www.irods.org}} \citep[see e.g.,][]{2007AGUFMIN13B1214R}
or a cloud? Similarly, we would
want FITS to contain, and adequately access, data generated by a
service (think simulation data, for example). Possibly FITS could
itself execute simple stored algorithms to generate a portion of its
data\footnote{This probably implies the need for a ``FITS language'' to generate these data.}.


These use cases are only examples of where we might go and some may be
arguably of limited value. Nevertheless, the generalized use case that
may be derived from all of these is of certain importance, namely,
A science data storage format should be able to support both local 
and remote data access, providing immediate and secure access to data 
contained within itself, and providing transparent access to data held 
in non-local entities such as cloud storage, databases, services, and 
other files.


The grouping convention includes a provision for specifying URIs,
but the implementation is not adequate. There is no provision for
directing the machine on how to access the resource specified by the
URI.


\section{Summary -- Significant problems exist in the FITS standard}


The problems which we have described are real and significant, however
we do not wish to recommend a particular solution here to any of
these. Action to correct these issues should flow from constructive
community discussion and offered solutions to these problems. Possible
solutions may involve moving existing FITS conventions into the core
standard, modification of the FITS standard to remove limitations or
possibly transferring the FITS data model over into a new
serialisation, or some selection of these actions.


These technical problems will be solved one way or another. If the
community is not willing to do the hard work of hammering out a
universal (or widely-adopted) approach, individual projects will
continue to make their own ad-hoc solutions. Data formats will become
increasingly fragmented and we will no longer enjoy the easy
interoperability that FITS has provided for many years.

\section{Acknowledgments}


MdVB acknowledges partial support from grants NSF AST-1108686 and NASA
NNX12AH91H.


%% The Appendices part is started with the command \appendix;
%% appendix sections are then done as normal sections
%% \appendix

%% \section{}
%% \label{}

%% References
%%
%% Following citation commands can be used in the body text:
%%
%%  \citet{key}  ==>>  Jones et al. (1990)
%%  \citep{key}  ==>>  (Jones et al., 1990)
%%
%% Multiple citations as normal:
%% \citep{key1,key2}         ==>> (Jones et al., 1990; Smith, 1989)
%%                            or  (Jones et al., 1990, 1991)
%%                            or  (Jones et al., 1990a,b)
%% \cite{key} is the equivalent of \citet{key} in author-year mode
%%
%% Full author lists may be forced with \citet* or \citep*, e.g.
%%   \citep*{key}            ==>> (Jones, Baker, and Williams, 1990)
%%
%% Optional notes as:
%%   \citep[chap. 2]{key}    ==>> (Jones et al., 1990, chap. 2)
%%   \citep[e.g.,][]{key}    ==>> (e.g., Jones et al., 1990)
%%   \citep[see][pg. 34]{key}==>> (see Jones et al., 1990, pg. 34)
%%  (Note: in standard LaTeX, only one note is allowed, after the ref.
%%   Here, one note is like the standard, two make pre- and post-notes.)
%%
%%   \citealt{key}          ==>> Jones et al. 1990
%%   \citealt*{key}         ==>> Jones, Baker, and Williams 1990
%%   \citealp{key}          ==>> Jones et al., 1990
%%   \citealp*{key}         ==>> Jones, Baker, and Williams, 1990
%%
%% Additional citation possibilities
%%   \citeauthor{key}       ==>> Jones et al.
%%   \citeauthor*{key}      ==>> Jones, Baker, and Williams
%%   \citeyear{key}         ==>> 1990
%%   \citeyearpar{key}      ==>> (1990)
%%   \citetext{priv. comm.} ==>> (priv. comm.)
%%   \citenum{key}          ==>> 11 [non-superscripted]
%% Note: full author lists depends on whether the bib style supports them;
%%       if not, the abbreviated list is printed even when full requested.
%%
%% For names like della Robbia at the start of a sentence, use
%%   \Citet{dRob98}         ==>> Della Robbia (1998)
%%   \Citep{dRob98}         ==>> (Della Robbia, 1998)
%%   \Citeauthor{dRob98}    ==>> Della Robbia


%% References with bibTeX database:

\bibliographystyle{model2-names}
\bibliography{acfits}

%% Authors are advised to submit their bibtex database files. They are
%% requested to list a bibtex style file in the manuscript if they do
%% not want to use model2-names.bst.

%% References without bibTeX database:

% \begin{thebibliography}{00}

%% \bibitem must have one of the following forms:
%%   \bibitem[Jones et al.(1990)]{key}...
%%   \bibitem[Jones et al.(1990)Jones, Baker, and Williams]{key}...
%%   \bibitem[Jones et al., 1990]{key}...
%%   \bibitem[\protect\citeauthoryear{Jones, Baker, and Williams}{Jones
%%       et al.}{1990}]{key}...
%%   \bibitem[\protect\citeauthoryear{Jones et al.}{1990}]{key}...
%%   \bibitem[\protect\astroncite{Jones et al.}{1990}]{key}...
%%   \bibitem[\protect\citename{Jones et al., }1990]{key}...
%%   \harvarditem[Jones et al.]{Jones, Baker, and Williams}{1990}{key}...
%%

% \bibitem[ ()]{}

% \end{thebibliography}

\end{document}

%%
%% End of file `elsarticle-template-2-harv.tex'.
